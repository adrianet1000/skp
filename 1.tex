\section{color del rectangulo segun el plano actual al momento de dibujar}

Cuando activamos la herramienta rectangulo, el rectangulo se pinta de color
al eje perpendicular al que estamos dibujando.

\section{La selección en sketchup igual a autocad}

Cuando seleccionamos de izquierda a derecha se necesita que los objetos
esten seleccionados totalmente para ser seleccionados, mientras que la
selección de derecha a izquierda se necesita que los objetos esten
seleccionados parcialmente para ser seleccionados.

Notar que esta selección se hace manteniendo el boton izquierdo del mouse
presionado.

\section{herramienta rectangulo invocado con la letra r}

Podemos modificar las dimensiones del rectangulo luego de crearla, sin
ejecutar un comando nuevamente, es decir escribiendo inmediatamente las
nuevas dimensiones del rectangulo, teniendo en cuenta de la configuración
de los separadores de decimales y los separadores de argumentos que se tiene
en el sketchup como lo es el punto y la coma por defecto.

Una vez que se haya escrito las dimensiones del rectangulo se debe presionar
intro para haceptar los cambios.

\section{tres formas de seleccinar un volumen}

un solo clic sobre una cara, solo seccionara la cara.

dos clics sobre una cara, seleccionara la cara y las aristas que componen esa
cara.

tres clics sobre una cara, y selecciona todo el volumen.

\section{atajos de teclado (shortcut)}

Los atajos de teclado podemos almacenarlos en un archivo de texto con la
extension .dat que para iniciar podemos exportar un archivo .dat con la
configuración por defecto que tiene sketchup 2021 pro, que yo le voy a poner
con el nombre shortcutDefaultSkp2021pro.dat , esta opción la encontramos en
el menu windows/preferences/shortcuts
